%%%%%%%%%%%%%%%%%%%%%%%%%%%%%%%%%%%%%%%%%%%%%%%%%%%%%%%%%%%%%%%
%
% Welcome to Overleaf --- just edit your LaTeX on the left,
% and we'll compile it for you on the right. If you open the
% 'Share' menu, you can invite other users to edit at the same
% time. See www.overleaf.com/learn for more info. Enjoy!
%
%%%%%%%%%%%%%%%%%%%%%%%%%%%%%%%%%%%%%%%%%%%%%%%%%%%%%%%%%%%%%%%

% Inbuilt themes in beamer
\documentclass{beamer}

%packages:
% \usepackage{tfrupee}
% \usepackage{amsmath}
% \usepackage{amssymb}
% \usepackage{gensymb}
% \usepackage{txfonts}

\def\inputGnumericTable{}

% \usepackage[latin1]{inputenc}                                 
% \usepackage{color}                                            
% \usepackage{array}                                            
% \usepackage{longtable}                                        
% \usepackage{calc}                                             
% \usepackage{multirow}                                         
% \usepackage{hhline}                                           
% \usepackage{ifthen}
% \usepackage{caption} 
% \captionsetup[table]{skip=3pt}  
 \providecommand{\pr}[1]{\ensuremath{\Pr\left(#1\right)}}
 \providecommand{\cbrak}[1]{\ensuremath{\left\{#1\right\}}}
% %\renewcommand{\thefigure}{\arabic{table}}
% \renewcommand{\thetable}{\arabic{table}}      

\setbeamertemplate{caption}[numbered]{}

\usepackage{enumitem}
\usepackage{tfrupee}
\usepackage{amsmath}
\usepackage{amssymb}
\usepackage{gensymb}
\usepackage{graphicx}
\usepackage{txfonts}

\def\inputGnumericTable{}

\usepackage[latin1]{inputenc}                                 
\usepackage{color}                                            
\usepackage{array}                                            
\usepackage{longtable}                                        
\usepackage{calc}                                             
\usepackage{multirow}                                         
\usepackage{hhline}                                           
\usepackage{ifthen}
\usepackage{caption} 
\captionsetup[table]{skip=3pt}  
\providecommand{\pr}[1]{\ensuremath{\Pr\left(#1\right)}}
\providecommand{\cbrak}[1]{\ensuremath{\left\{#1\right\}}}
\renewcommand{\thefigure}{\arabic{table}}
\renewcommand{\thetable}{\arabic{table}}   
\providecommand{\brak}[1]{\ensuremath{\left(#1\right)}}

% Theme choice:
\usetheme{CambridgeUS}

% Title page details: 
\title{Assignment 8} 
\author{Manikanta Uppulapu}
\date{\today}
% \logo{\large \LaTeX{}}


\begin{document}

% Title page frame
\begin{frame}
    \titlepage 
\end{frame}

% Remove logo from the next slides
\logo{}


% Outline frame
\begin{frame}{Outline}
    \tableofcontents
\end{frame}



\section{Problem Statement}
\begin{frame}{Problem Statement}
    \begin{block} {QUESTION}A player wins \$1 if he throws two heads in succession, otherwise he loses two quarters.
If the game is repeated 50 times, what is the probability that the net gain or loss exceeds\\
(a) \$1 (b) \$5
    \end{block}
\end{frame}



\section{Given}
\begin{frame}{Given}
Let's denote the situation of the problem by random variable's $X$ and $Y$  such that $X\in \cbrak{0,1}$,$Y\in \cbrak{0,1}$\\
where,\\
\begin{table}[ht!]
    \centering
    \input{table}
    \caption{}
\label{table:table1}
\end{table}

\end{frame}

\begin{frame}
	Consider an experiment consisting of $50$ Bernoulli trials and denote the number of two heads in succession obtained by a binomial random variable $Y \in \cbrak{0,1,\ldots,50}$. This can be expressed as a binomial distribution with probability mass function given by:
	\begin{align}
		p_Y(k) = \binom{n}{k} (1-p)^{n-k} p^k,~ 0 \le k \le n
	\end{align}
	where $n = 50$ and $p = 0.25$
	\end{frame}
	
\section{Solution}
\begin{frame}{Solution}
(a) Let k represent the number of wins required in 50 games so that
the net gain or loss does not exceed \$1. This gives the net gain to be 
\begin{align}
    -1 &< k - \frac{50-k}{2} <  1\\
     16 &< k < 17.3 \\
      k &= 17
      \end{align}
      The desired probability is given by:
	\begin{align}
		p_{X=0}(17)&= \binom{50}{17} (1-0.25)^{33} (0.25)^{17} \\
		&=0.432
\end{align}
\therefore \pr{X=1} = 1-0.432 = 0.568
\end{frame}

\begin{frame}{Solution}
(a) Let k represent the number of wins required in 50 games so that
the net gain or loss does not exceed \$5. This gives 
\begin{align}
    -5 &< k - \frac{50-k}{2} <  5\\
     13.3 &< k < 20 
      \end{align}
      The desired probability is given by:
	\begin{align}
		p_{Y=0}(k)&=\sum_{k=14}^{19} \binom{50}{n} (1-0.25)^{n-k} (0.25)^{k}\\
		&=0.349
\end{align}
\therefore \pr{Y=1} = 1-0.349 = 0.651
\end{frame}
\end{document}
